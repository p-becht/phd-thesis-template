%!TeX root = ../main.tex
%
\section{DPTS testbeam campaigns}
\label{sec:testbeams}
%

%Furthermore, constraints on performance quantities like the noise occupancy or hit detection efficiency as detailed in the chapters \ref{chap:dpts-results} and \ref{chap:babymoss-results} are considered in order to define a suitable operating range for the respective sensor, which also incorporates the requirements for the final application scenario such as the \gls{its3}. 

\begin{figure}[!htb]
	\centering
    %!TeX root = ./testbeam_setups.tex

\begin{scaletikzpicturetowidth}{\textwidth}
    \begin{tikzpicture}[decoration=brace, global scale = \tikzscale]
			\tikzmath{
				% general distances
				\PlaneDistanceScale = 0.75;
                \DutHeight = 1.5;
                \DPTSHeight = 0.05 * 10;
				\BoardHeight = 1.0;
				\LabelScale=1.5;
				\BoardHeightmod = \BoardHeight * 0.85;
                			\BoardHeightDPTS = \BoardHeight + 0.5 * \DutHeight - 0.5 * \DPTSHeight;
				\BoardHeightmodDPTS = \BoardHeightDPTS * 0.85;
				\BoardWidth = 0.04;
				\ChipWidth = 0.5mm;
				\CoordinateDist = 0.3;
				\ArrWidth = 1.75mm;
				\ArrLength = \ArrWidth * 1.2;
				\CoordinateLength = 1;
				\VecRad = 0.15;
				\SciFullHeight = 2;
				\SciHoleHeight = \DPTSHeight;
				\SciWidth = \BoardWidth * 3;
				\SciHeight = \SciFullHeight / 2 - \SciHoleHeight / 2;
				%
				% distances between planes
				\z01 = 2.5 * \PlaneDistanceScale;
				\z12 = 2.5 * \PlaneDistanceScale;
				\z23 = 3.0 * \PlaneDistanceScale;
				\z34 = 3.0 * \PlaneDistanceScale;
				\z45 = 4.0 * \PlaneDistanceScale;
				\z56 = 2.5 * \PlaneDistanceScale;
				\z67 = 2.5 * \PlaneDistanceScale;
				\zSci = 2.5 * \PlaneDistanceScale;
				%
				\px0 = 0;
				\pylow = \DutHeight / -2;
				\pyup = \DutHeight / 2;
                			\pylowdpts = \DPTSHeight / -2;
				\pyupdpts = \DPTSHeight / 2;
				\pylowsci = \SciHoleHeight / -2;
				\pyupsci = \SciHoleHeight / 2;
				%
				\px1 = \px0 + \z01;
				\px2 = \px1 + \z12;
				\px3 = \px2 + \z23;
				\px4 = \px3 + \z34;
				\px5 = \px4 + \z45;
				\px6 = \px5 + \z56;
				\px7 = \px6 + \z67;
				%
				\pxS1 = \px0-\zSci-\zSci;
				\pxS2 = \px0-\zSci;
				\pxS3 = \px7+\zSci-\SciWidth;
			}
		
			% coordinate calculation
			\coordinate (p0lowUL) at (\px0,\pylow);
			\coordinate (p0upBL) at (\px0,\pyup);
			\coordinate (p1lowUL) at (\px1,\pylow);
			\coordinate (p1upBL) at (\px1,\pyup);
			\coordinate (p2lowUL) at (\px2,\pylow);
			\coordinate (p2upBL) at (\px2,\pyup);
			
          			\coordinate (p3lowUL) at (\px3,\pylowdpts);
			\coordinate (p3upBL) at (\px3,\pyupdpts);
		           \coordinate (p4lowUL) at (\px4,\pylowdpts);
			\coordinate (p4upBL) at (\px4,\pyupdpts);
			\coordinate (p3lowULref) at (\px3,\pylow);
			\coordinate (p3upBLref) at (\px3,\pyup);
		           \coordinate (p4lowULref) at (\px4,\pylow);
			\coordinate (p4upBLref) at (\px4,\pyup);
			
			\coordinate (p5lowUL) at (\px5,\pylow);
			\coordinate (p5upBL) at (\px5,\pyup);
			\coordinate (p6lowUL) at (\px6,\pylow);
			\coordinate (p6upBL) at (\px6,\pyup);
			\coordinate (p7lowUL) at (\px7,\pylow);
			\coordinate (p7upBL) at (\px7,\pyup);
			
			\coordinate (s1lowUL) at (\pxS1,\pylowsci);
			\coordinate (s1upBL) at (\pxS1,\pyupsci);
			\coordinate (s2lowUL) at (\pxS2,\pylowsci);
			\coordinate (s2upBL) at (\pxS2,\pyupsci);
			\coordinate (s3lowUL) at (\pxS3,\pylowsci);
			\coordinate (s3upBL) at (\pxS3,\pyupsci);
			
			% beam direction
			\draw[beamarrow, color=Blue, thick, dashed] ($(\pxS1,0) + (-0.5,0)$)
				--  ($(\px2 + \BoardWidth,0) $)
				-- ($(\px3,0) $)
				-- ($(\px4,0) $)
				--  ($(\px5,0) $)
				--  ($(\px6,0) $)
				--  node[below] {\textcolor{Blue}{direction}}  node[above] {\textcolor{Blue}{Beam}}($(\px7,0) $)
				-- ($(\pxS3 + \SciWidth + 0.6,0) $);
				
			% scintillators
			% S1
			\draw[fill=YellowOrange] (s1lowUL)
				-- ($(s1lowUL) + (\SciWidth,0)$)
				--  ($(s1lowUL) + (\SciWidth,-1 * \SciHeight)$)
				-- ($(s1lowUL) + (0,-1 * \SciHeight)$)
				-- cycle;
			\draw[fill=YellowOrange] (s1upBL)
				-- ($(s1upBL) + (\SciWidth,0)$)
				--  ($(s1upBL) + (\SciWidth,\SciHeight)$)
				-- ($(s1upBL) + (0,\SciHeight)$)
				-- cycle;
				
			%S2
			\draw[fill=YellowOrange]  ($(s2lowUL) + (0,-1 * \SciHeight)$)
				-- ($(s2lowUL) + (\SciWidth,-1 * \SciHeight)$)
				--  ($(s2upBL) + (\SciWidth,\SciHeight)$)
				-- ($(s2upBL) + (0,\SciHeight)$)
				-- cycle;
				
			%S3
			\draw[fill=YellowOrange]  ($(s3lowUL) + (0,-1 * \SciHeight)$)
				-- ($(s3lowUL) + (\SciWidth,-1 * \SciHeight)$)
				--  ($(s3upBL) + (\SciWidth,\SciHeight)$)
				-- ($(s3upBL) + (0,\SciHeight)$)
				-- cycle;
			
			% chips
			\draw[line width = \ChipWidth, color=BrickRed] ($(p0upBL) + (0,0)$)
				--node[right, yshift=0.02\textwidth, xshift=-0.005\textwidth] {\large\textcolor{BrickRed}{ALPIDE}}
				($(p0upBL) + (0,-1 * \DutHeight)$);
			\draw[line width = \ChipWidth, color=BrickRed] ($(p1upBL) + (0,0)$)
				-- ($(p1upBL) + (0,-1 * \DutHeight)$);
			\draw[line width = \ChipWidth, color=BrickRed] ($(p2upBL) + (0,0)$)
				-- ($(p2upBL) + (0,-1 * \DutHeight)$);
				
			\draw[line width = \ChipWidth, color=BrickRed] ($(p3upBL) + (0,0)$)
                			--node[right, yshift=0.02\textwidth] {\large\textcolor{BrickRed}{DPTS}}
				($(p3upBL) + (0,-1 * \DPTSHeight)$);
            		\draw[line width = \ChipWidth, color=BrickRed] ($(p4upBL) + (0,0)$)
				-- ($(p4upBL) + (0,-1 * \DPTSHeight)$);				

			\draw[line width = \ChipWidth, color=BrickRed] ($(p5upBL) + (0,0)$)
				-- ($(p5upBL) + (0,-1 * \DutHeight)$);
			\draw[line width = \ChipWidth, color=BrickRed] ($(p6upBL) + (0,0)$)
				-- ($(p6upBL) + (0,-1 * \DutHeight)$);
			\draw[line width = \ChipWidth, color=BrickRed] ($(p7upBL) + (0,0)$)
				-- ($(p7upBL) + (0,-1 * \DutHeight)$);
				
			% carrier boards
			% plane 0
			\fill[top color = gray!100,bottom color = gray!25, middle color=gray] (p0lowUL)
				-- ($(p0lowUL) + (\BoardWidth,0)$)
				--  ($(p0lowUL) + (\BoardWidth,-1 * \BoardHeight)$)
				-- ($(p0lowUL) + (0,-1 * \BoardHeight)$)
				-- cycle; % fading colour rectangle low
			\draw [] (p0lowUL)
				edge [densely dotted]  ($(p0lowUL) + (0,-1 * \BoardHeight)$); % left side low
			\draw [] ($(p0lowUL) + (\BoardWidth,0)$)
				edge [densely dotted]  ($(p0lowUL) + (\BoardWidth,-1 * \BoardHeight)$); % right side low
			\draw [] ($(p0lowUL) + (0,-1 * \BoardHeightmod)$)
				-- (p0lowUL)
				-- ($(p0lowUL) + (\BoardWidth,0)$)
				-- ($(p0lowUL) + (\BoardWidth,-1 * \BoardHeightmod)$); % top side low

			\fill[bottom color = gray!100,top color = gray!25, middle color=gray] (p0upBL)
				-- ($(p0upBL) + (\BoardWidth,0)$)
				--  ($(p0upBL) + (\BoardWidth,\BoardHeight)$)
				-- ($(p0upBL) + (0,\BoardHeight)$)
				-- cycle; % fading colour rectangle up
			\draw [] (p0upBL)
				edge [densely dotted]  ($(p0upBL) + (0,\BoardHeight)$); % left side up
			\draw [] ($(p0upBL) + (\BoardWidth,0)$)
				edge [densely dotted]  ($(p0upBL) + (\BoardWidth,\BoardHeight)$); % right side up
			\draw [] ($(p0upBL) + (0,\BoardHeightmod)$)
				-- (p0upBL)
				-- ($(p0upBL) + (\BoardWidth,0)$)
				-- ($(p0upBL) + (\BoardWidth, \BoardHeightmod)$); % top side up

			% plane 1
			\fill[top color = gray!100,bottom color = gray!25, middle color=gray] (p1lowUL)
				-- ($(p1lowUL) + (\BoardWidth,0)$)
				--  ($(p1lowUL) + (\BoardWidth,-1 * \BoardHeight)$)
				-- ($(p1lowUL) + (0,-1 * \BoardHeight)$)
				-- cycle; % fading colour rectangle low
			\draw [] (p1lowUL)
				edge [densely dotted]  ($(p1lowUL) + (0,-1 * \BoardHeight)$); % left side low
			\draw [] ($(p1lowUL) + (\BoardWidth,0)$)
				edge [densely dotted]  ($(p1lowUL) + (\BoardWidth,-1 * \BoardHeight)$); % right side low
			\draw [] ($(p1lowUL) + (0,-1 * \BoardHeightmod)$)
				-- (p1lowUL)
				-- ($(p1lowUL) + (\BoardWidth,0)$)
				-- ($(p1lowUL) + (\BoardWidth,-1 * \BoardHeightmod)$); % top side low

			\fill[bottom color = gray!100,top color = gray!25, middle color=gray] (p1upBL)
				-- ($(p1upBL) + (\BoardWidth,0)$)
				--  ($(p1upBL) + (\BoardWidth,\BoardHeight)$)
				-- ($(p1upBL) + (0,\BoardHeight)$)
				-- cycle; % fading colour rectangle up
			\draw [] (p1upBL)
				edge [densely dotted]  ($(p1upBL) + (0,\BoardHeight)$); % left side up
			\draw [] ($(p1upBL) + (\BoardWidth,0)$)
				edge [densely dotted]  ($(p1upBL) + (\BoardWidth,\BoardHeight)$); % right side up
			\draw [] ($(p1upBL) + (0,\BoardHeightmod)$)
				-- (p1upBL)
				-- ($(p1upBL) + (\BoardWidth,0)$)
				-- ($(p1upBL) + (\BoardWidth, \BoardHeightmod)$); % top side up
				
			% plane 2
			\fill[top color = gray!100,bottom color = gray!25, middle color=gray] (p2lowUL)
				-- ($(p2lowUL) + (\BoardWidth,0)$)
				--  ($(p2lowUL) + (\BoardWidth,-1 * \BoardHeight)$)
				-- ($(p2lowUL) + (0,-1 * \BoardHeight)$)
				-- cycle; % fading colour rectangle low
			\draw [] (p2lowUL)
				edge [densely dotted]  ($(p2lowUL) + (0,-1 * \BoardHeight)$); % left side low
			\draw [] ($(p2lowUL) + (\BoardWidth,0)$)
				edge [densely dotted]  ($(p2lowUL) + (\BoardWidth,-1 * \BoardHeight)$); % right side low
			\draw [] ($(p2lowUL) + (0,-1 * \BoardHeightmod)$)
				-- (p2lowUL)
				-- ($(p2lowUL) + (\BoardWidth,0)$)
				-- ($(p2lowUL) + (\BoardWidth,-1 * \BoardHeightmod)$); % top side low

			\fill[bottom color = gray!100,top color = gray!25, middle color=gray] (p2upBL)
				-- ($(p2upBL) + (\BoardWidth,0)$)
				--  ($(p2upBL) + (\BoardWidth,\BoardHeight)$)
				-- ($(p2upBL) + (0,\BoardHeight)$)
				-- cycle; % fading colour rectangle up
			\draw [] (p2upBL)
				edge [densely dotted]  ($(p2upBL) + (0,\BoardHeight)$); % left side up
			\draw [] ($(p2upBL) + (\BoardWidth,0)$)
				edge [densely dotted]  ($(p2upBL) + (\BoardWidth,\BoardHeight)$); % right side up
			\draw [] ($(p2upBL) + (0,\BoardHeightmod)$)
				-- (p2upBL)
				-- ($(p2upBL) + (\BoardWidth,0)$)
				-- ($(p2upBL) + (\BoardWidth, \BoardHeightmod)$); % top side up

			% plane 3
			\fill[top color = gray!100,bottom color = gray!25, middle color=gray] (p3lowUL)
				-- ($(p3lowUL) + (\BoardWidth,0)$)
				--  ($(p3lowUL) + (\BoardWidth,-1 * \BoardHeightDPTS)$)
				-- ($(p3lowUL) + (0,-1 * \BoardHeightDPTS)$)
				-- cycle; % fading colour rectangle low
			\draw [] (p3lowUL)
				edge [densely dotted]  ($(p3lowUL) + (0,-1 * \BoardHeightDPTS)$); % left side low
			\draw [] ($(p3lowUL) + (\BoardWidth,0)$)
				edge [densely dotted]  ($(p3lowUL) + (\BoardWidth,-1 * \BoardHeightDPTS)$); % right side low
			\draw [] ($(p3lowUL) + (0,-1 * \BoardHeightmodDPTS)$)
				-- (p3lowUL)
				-- ($(p3lowUL) + (\BoardWidth,0)$)
				-- ($(p3lowUL) + (\BoardWidth,-1 * \BoardHeightmodDPTS)$); % top side low

			\fill[bottom color = gray!100,top color = gray!25, middle color=gray] (p3upBL)
				-- ($(p3upBL) + (\BoardWidth,0)$)
				--  ($(p3upBL) + (\BoardWidth,\BoardHeightDPTS)$)
				-- ($(p3upBL) + (0,\BoardHeightDPTS)$)
				-- cycle; % fading colour rectangle up
			\draw [] (p3upBL)
				edge [densely dotted]  ($(p3upBL) + (0,\BoardHeightDPTS)$); % left side up
			\draw [] ($(p3upBL) + (\BoardWidth,0)$)
				edge [densely dotted]  ($(p3upBL) + (\BoardWidth,\BoardHeightDPTS)$); % right side up
			\draw [] ($(p3upBL) + (0,\BoardHeightmodDPTS)$)
				-- (p3upBL)
				-- ($(p3upBL) + (\BoardWidth,0)$)
				-- ($(p3upBL) + (\BoardWidth, \BoardHeightmodDPTS)$); % top side up

			% plane 4
			\fill[top color = gray!100,bottom color = gray!25, middle color=gray] (p4lowUL)
				-- ($(p4lowUL) + (\BoardWidth,0)$)
				--  ($(p4lowUL) + (\BoardWidth,-1 * \BoardHeightDPTS)$)
				-- ($(p4lowUL) + (0,-1 * \BoardHeightDPTS)$)
				-- cycle; % fading colour rectangle low
			\draw [] (p4lowUL)
				edge [densely dotted]  ($(p4lowUL) + (0,-1 * \BoardHeightDPTS)$); % left side low
			\draw [] ($(p4lowUL) + (\BoardWidth,0)$)
				edge [densely dotted]  ($(p4lowUL) + (\BoardWidth,-1 * \BoardHeightDPTS)$); % right side low
			\draw [] ($(p4lowUL) + (0,-1 * \BoardHeightmodDPTS)$)
				-- (p4lowUL)
				-- ($(p4lowUL) + (\BoardWidth,0)$)
				-- ($(p4lowUL) + (\BoardWidth,-1 * \BoardHeightmodDPTS)$); % top side low

			\fill[bottom color = gray!100,top color = gray!25, middle color=gray] (p4upBL)
				-- ($(p4upBL) + (\BoardWidth,0)$)
				--  ($(p4upBL) + (\BoardWidth,\BoardHeightDPTS)$)
				-- ($(p4upBL) + (0,\BoardHeightDPTS)$)
				-- cycle; % fading colour rectangle up
			\draw [] (p4upBL)
				edge [densely dotted]  ($(p4upBL) + (0,\BoardHeightDPTS)$); % left side up
			\draw [] ($(p4upBL) + (\BoardWidth,0)$)
				edge [densely dotted]  ($(p4upBL) + (\BoardWidth,\BoardHeightDPTS)$); % right side up
			\draw [] ($(p4upBL) + (0,\BoardHeightmodDPTS)$)
				-- (p4upBL)
				-- ($(p4upBL) + (\BoardWidth,0)$)
				-- ($(p4upBL) + (\BoardWidth, \BoardHeightmodDPTS)$); % top side up

			% plane 5
			\fill[top color = gray!100,bottom color = gray!25, middle color=gray] (p5lowUL)
				-- ($(p5lowUL) + (\BoardWidth,0)$)
				--  ($(p5lowUL) + (\BoardWidth,-1 * \BoardHeight)$)
				-- ($(p5lowUL) + (0,-1 * \BoardHeight)$)
				-- cycle; % fading colour rectangle low
			\draw [] (p5lowUL)
				edge [densely dotted]  ($(p5lowUL) + (0,-1 * \BoardHeight)$); % left side low
			\draw [] ($(p5lowUL) + (\BoardWidth,0)$)
				edge [densely dotted]  ($(p5lowUL) + (\BoardWidth,-1 * \BoardHeight)$); % right side low
			\draw [] ($(p5lowUL) + (0,-1 * \BoardHeightmod)$)
				-- (p5lowUL)
				-- ($(p5lowUL) + (\BoardWidth,0)$)
				-- ($(p5lowUL) + (\BoardWidth,-1 * \BoardHeightmod)$); % top side low

			\fill[bottom color = gray!100,top color = gray!25, middle color=gray] (p5upBL)
				-- ($(p5upBL) + (\BoardWidth,0)$)
				--  ($(p5upBL) + (\BoardWidth,\BoardHeight)$)
				-- ($(p5upBL) + (0,\BoardHeight)$)
				-- cycle; % fading colour rectangle up
			\draw [] (p5upBL)
				edge [densely dotted]  ($(p5upBL) + (0,\BoardHeight)$); % left side up
			\draw [] ($(p5upBL) + (\BoardWidth,0)$)
				edge [densely dotted]  ($(p5upBL) + (\BoardWidth,\BoardHeight)$); % right side up
			\draw [] ($(p5upBL) + (0,\BoardHeightmod)$)
				-- (p5upBL)
				-- ($(p5upBL) + (\BoardWidth,0)$)
				-- ($(p5upBL) + (\BoardWidth, \BoardHeightmod)$); % top side up

			% plane 6
			\fill[top color = gray!100,bottom color = gray!25, middle color=gray] (p6lowUL)
				-- ($(p6lowUL) + (\BoardWidth,0)$)
				--  ($(p6lowUL) + (\BoardWidth,-1 * \BoardHeight)$)
				-- ($(p6lowUL) + (0,-1 * \BoardHeight)$)
				-- cycle; % fading colour rectangle low
			\draw [] (p6lowUL)
				edge [densely dotted]  ($(p6lowUL) + (0,-1 * \BoardHeight)$); % left side low
			\draw [] ($(p6lowUL) + (\BoardWidth,0)$)
				edge [densely dotted]  ($(p6lowUL) + (\BoardWidth,-1 * \BoardHeight)$); % right side low
			\draw [] ($(p6lowUL) + (0,-1 * \BoardHeightmod)$)
				-- (p6lowUL)
				-- ($(p6lowUL) + (\BoardWidth,0)$)
				-- ($(p6lowUL) + (\BoardWidth,-1 * \BoardHeightmod)$); % top side low

			\fill[bottom color = gray!100,top color = gray!25, middle color=gray] (p6upBL)
				-- ($(p6upBL) + (\BoardWidth,0)$)
				--  ($(p6upBL) + (\BoardWidth,\BoardHeight)$)
				-- ($(p6upBL) + (0,\BoardHeight)$)
				-- cycle; % fading colour rectangle up
			\draw [] (p6upBL)
				edge [densely dotted]  ($(p6upBL) + (0,\BoardHeight)$); % left side up
			\draw [] ($(p6upBL) + (\BoardWidth,0)$)
				edge [densely dotted]  ($(p6upBL) + (\BoardWidth,\BoardHeight)$); % right side up
			\draw [] ($(p6upBL) + (0,\BoardHeightmod)$)
				-- (p6upBL)
				-- ($(p6upBL) + (\BoardWidth,0)$)
				-- ($(p6upBL) + (\BoardWidth, \BoardHeightmod)$); % top side up
				
			% plane 7
			\fill[top color = gray!100,bottom color = gray!25, middle color=gray] (p7lowUL)
				-- ($(p7lowUL) + (\BoardWidth,0)$)
				--  ($(p7lowUL) + (\BoardWidth,-1 * \BoardHeight)$)
				-- ($(p7lowUL) + (0,-1 * \BoardHeight)$)
				-- cycle; % fading colour rectangle low
			\draw [] (p7lowUL)
				edge [densely dotted]  ($(p7lowUL) + (0,-1 * \BoardHeight)$); % left side low
			\draw [] ($(p7lowUL) + (\BoardWidth,0)$)
				edge [densely dotted]  ($(p7lowUL) + (\BoardWidth,-1 * \BoardHeight)$); % right side low
			\draw [] ($(p7lowUL) + (0,-1 * \BoardHeightmod)$)
				-- (p7lowUL)
				-- ($(p7lowUL) + (\BoardWidth,0)$)
				-- ($(p7lowUL) + (\BoardWidth,-1 * \BoardHeightmod)$); % top side low

			\fill[bottom color = gray!100,top color = gray!25, middle color=gray] (p7upBL)
				-- ($(p7upBL) + (\BoardWidth,0)$)
				--  ($(p7upBL) + (\BoardWidth,\BoardHeight)$)
				-- ($(p7upBL) + (0,\BoardHeight)$)
				-- cycle; % fading colour rectangle up
			\draw [] (p7upBL)
				edge [densely dotted]  ($(p7upBL) + (0,\BoardHeight)$); % left side up
			\draw [] ($(p7upBL) + (\BoardWidth,0)$)
				edge [densely dotted]  ($(p7upBL) + (\BoardWidth,\BoardHeight)$); % right side up
			\draw [] ($(p7upBL) + (0,\BoardHeightmod)$)
				-- (p7upBL)
				-- ($(p7upBL) + (\BoardWidth,0)$)
				-- ($(p7upBL) + (\BoardWidth, \BoardHeightmod)$); % top side up

			% board labels
			\node[above, align=center] at ($(p0upBL) + (\BoardWidth / 2, \BoardHeight)$) {\large 0};
			\node[above, align=center] at ($(p1upBL) + (\BoardWidth / 2, \BoardHeight)$) {\large 1};
			\node[above, align=center] at ($(p2upBL) + (\BoardWidth / 2, \BoardHeight)$) {\large 2};
			\node[above, align=center] at ($(p3upBLref) + (\BoardWidth / 2, \BoardHeight)$) {\large 3};
			\node[above, align=center] at ($(p4upBLref) + (\BoardWidth / 2, \BoardHeight)$) {\large 4};
			\node[above, align=center] at ($(p5upBL) + (\BoardWidth / 2, \BoardHeight)$) {\large 5};
			\node[above, align=center] at ($(p6upBL) + (\BoardWidth / 2, \BoardHeight)$) {\large 6};
			\node[above, align=center] at ($(p7upBL) + (\BoardWidth / 2, \BoardHeight)$) {\large 7};
			
			\node[above, align=center] at ($(s1upBL) + (\SciWidth / 2, \SciHeight)$) {\large S1};
			\node[above, align=center] at ($(s2upBL) + (\SciWidth / 2, \SciHeight)$) {\large S2};
			\node[above, align=center] at ($(s3upBL) + (\SciWidth / 2, \SciHeight)$) {\large S3};
			
			%\node[above=0.4ex, align=center] at ($(\pxS1,\pyup) + (\SciWidth / 2,\LabelScale * \BoardHeight)$) {\large Veto};
			%\node[above=0.4ex, align=center] at ($(\pxS2,\pyup) + (\SciWidth / 2,\LabelScale * \BoardHeight)$) {\large Scintillator};
			\node[above=0.4ex, align=center] at ($(\pxS2,\pyup) + (\SciWidth / 2 - 1 * \zSci/2,\LabelScale * \BoardHeight)$) {\large Scintillators};
			\node[above=0.4ex, align=center] at ($(\pxS3,\pyup) + (\SciWidth / 2,\LabelScale * \BoardHeight)$) {\large Scintillator};
			
			\draw[decorate, yshift=-4ex]  ($(p0upBL) + (0,\LabelScale * \BoardHeight)$)
			-- node[above=0.4ex] {\large Reference arm} ($(p2upBL) + (\BoardWidth,\LabelScale * \BoardHeight)$);
			\draw[decorate, yshift=-4ex]  ($(p5upBL) + (0,\LabelScale * \BoardHeight)$)
			-- node[above=0.4ex] {\large Reference arm} ($(p7upBL) + (\BoardWidth,\LabelScale * \BoardHeight)$);
			
            		\node[above=0.4ex, align=center] at ($(p3upBLref) + (\BoardWidth / 2,\LabelScale * \BoardHeight)$) {\large DPTS 1};
			\node[above=0.4ex, align=center] at ($(p4upBLref) + (\BoardWidth / 2,\LabelScale * \BoardHeight)$) {\large DPTS 2};
			
           			 % plane distance labels
			\draw[stealth-stealth]  ($(p0lowUL) + (\BoardWidth / 2,-1.1 * \BoardHeight)$)
			-- node[above] {\SI{25}{\milli\metre}} ($(p1lowUL) + (\BoardWidth / 2,-1.1 * \BoardHeight)$);
			\draw[stealth-stealth]  ($(p1lowUL) + (\BoardWidth / 2,-1.1 * \BoardHeight)$)
			-- node[above] {\SI{25}{\milli\metre}} ($(p2lowUL) + (\BoardWidth / 2,-1.1 * \BoardHeight)$);
			\draw[stealth-stealth]  ($(p2lowUL) + (\BoardWidth / 2,-1.1 * \BoardHeight)$)
			-- node[above] {$d_1$} ($(p3lowULref) + (\BoardWidth / 2,-1.1 * \BoardHeight)$);
			\draw[stealth-stealth]  ($(p3lowULref) + (\BoardWidth / 2,-1.1 * \BoardHeight)$)
			-- node[above] {$d_2$} ($(p4lowULref) + (\BoardWidth / 2,-1.1 * \BoardHeight)$);
			\draw[stealth-stealth]  ($(p4lowULref) + (\BoardWidth / 2,-1.1 * \BoardHeight)$)
			-- node[above] {$d_3$} ($(p5lowUL) + (\BoardWidth / 2,-1.1 * \BoardHeight)$);
			\draw[stealth-stealth]  ($(p5lowUL) + (\BoardWidth / 2,-1.1 * \BoardHeight)$)
			-- node[above] {\SI{25}{\milli\metre}} ($(p6lowUL) + (\BoardWidth / 2,-1.1 * \BoardHeight)$);
			\draw[stealth-stealth]  ($(p6lowUL) + (\BoardWidth / 2,-1.1 * \BoardHeight)$)
			-- node[above] {\SI{25}{\milli\metre}} ($(p7lowUL) + (\BoardWidth / 2,-1.1 * \BoardHeight)$);
			
			% x direction (local coordinates)
			%\path  ($(p2upBL) + (\CoordinateDist,-1 * \DutHeight)$)
			%	pic {vector out};
			%\node[right, align=center] at  ($(p2upBL) + (\CoordinateDist + \VecRad,-1 * \DutHeight)$) {\large Column};
			% y direction (local coordinates)
			%\draw[densely dotted, thick] ($(p2upBL) + (\CoordinateDist,-1 * \DutHeight)$)
			%	-- ($(p2upBL) + (\CoordinateDist,-1 * \DutHeight + \VecRad)$);
			%\draw[-{Latex[width=\ArrWidth, length=\ArrLength]}, thick] ($(p2upBL) + (\CoordinateDist,-1 * \DutHeight + \VecRad)$)
			%	-- ($(p2upBL) + (\CoordinateDist,0)$);
			%\node[right, align=center] at   ($(p2upBL) + (\CoordinateDist + \VecRad,0)$) {\large Row};
			
			% x direction (global coordinates)
			\path  ($(p4upBLref) + (\CoordinateDist,-1 * \DutHeight / 2)$)
				pic {vector out};
			\node[below, align=center, xshift = 1ex] at  ($(p4upBLref) + (\CoordinateDist,-1 * \DutHeight / 2 - \VecRad)$) {\large $x$};
			% y direction (global coordinates)
			\draw[densely dotted, thick] ($(p4upBLref) + (\CoordinateDist,-1 * \DutHeight / 2)$)
				-- ($(p4upBLref) + (\CoordinateDist,-1 * \DutHeight / 2 + \VecRad)$);
			\draw[-{Latex[width=\ArrWidth, length=\ArrLength]}, thick] ($(p4upBLref) + (\CoordinateDist,-1 * \DutHeight / 2 + \VecRad)$)
				-- ($(p4upBLref) + (\CoordinateDist,-1 * \DutHeight / 2 + \VecRad + \CoordinateLength)$);
			\node[above, align=center, xshift = 1ex] at  ($(p4upBLref) + (\CoordinateDist,-1 * \DutHeight / 2 + \VecRad + \CoordinateLength)$) {\large $y$};
			% z direction (global coordinates)
			\draw[densely dotted, thick] ($(p4upBLref) + (\CoordinateDist,-1 * \DutHeight / 2)$)
				-- ($(p4upBLref) + (\CoordinateDist + \VecRad,-1 * \DutHeight / 2)$);
			\draw[-{Latex[width=\ArrWidth, length=\ArrLength]}, thick] ($(p4upBLref) + (\CoordinateDist + \VecRad,-1 * \DutHeight / 2)$)
				-- ($(p4upBLref) + (\CoordinateDist + \CoordinateLength,-1 * \DutHeight / 2)$);
			\node[above, align=center] at ($(p4upBLref) + (\CoordinateDist + \CoordinateLength,-1 * \DutHeight / 2)$) {\large $z$};
			
			% absolute global z-axis
			%\draw[-{Latex[width=\ArrWidth, length=\ArrLength]}, thick] ($(p0lowUL) + (-0.5,-1.4 * \BoardHeight)$)
			%	-- ($(p7lowUL) + (0.5,-1.4 * \BoardHeight)$);
			%\node[above, align=center] at  ($(p7lowUL) + (0.5,-1.4 * \BoardHeight)$) {\large $z$};
			
			% -107.5mm, -82.5mm, -57.5mm, -1.43231042mm, 17mm, 41.5mm, 64.5mm
			%\draw ($(p0lowUL) + (0,-1.3 * \BoardHeight)$) -- ($(p0lowUL) + (0,-1.5 * \BoardHeight)$) node[below=4pt] {\SI{-110}{\mm}};
			%\draw ($(p1lowUL) + (0,-1.3 * \BoardHeight)$) -- ($(p1lowUL) + (0,-1.5 * \BoardHeight)$) node[below=4pt] {\SI{-85}{\mm}};
			%\draw ($(p2lowUL) + (0,-1.3 * \BoardHeight)$) -- ($(p2lowUL) + (0,-1.5 * \BoardHeight)$) node[below=4pt] {\SI{-60}{\mm}};
			%\draw ($(p3lowULref) + (0,-1.3 * \BoardHeight)$) -- ($(p3lowULref) + (0,-1.5 * \BoardHeight)$) node[below=4pt] {\SI{-30}{\mm}};
			%\draw ($(p4lowULref) + (0,-1.3 * \BoardHeight)$) -- ($(p4lowULref) + (0,-1.5 * \BoardHeight)$) node[below=4pt] {\SI{0}{\mm}};
			%\draw ($(p5lowUL) + (0,-1.3 * \BoardHeight)$) -- ($(p5lowUL) + (0,-1.5 * \BoardHeight)$) node[below=4pt] {\SI{40}{\mm}};
			%\draw ($(p6lowUL) + (0,-1.3 * \BoardHeight)$) -- ($(p6lowUL) + (0,-1.5 * \BoardHeight)$) node[below=4pt] {\SI{65}{\mm}};
			%\draw ($(p7lowUL) + (0,-1.3 * \BoardHeight)$) -- ($(p7lowUL) + (0,-1.5 * \BoardHeight)$) node[below=4pt] {\SI{90}{\mm}};
	\end{tikzpicture}
\end{scaletikzpicturetowidth}
    \caption[Exemplary sketch of a beam telescope as used for DPTS testbeam campaigns]{Exemplary sketch of a beam telescope (not to scale) as used for DPTS testbeam campaigns. The DPTS sensors act as \gls{dut}. They are placed in between reference planes equipped with ALPIDE sensors. The scintillators (S2, S3) are usually operated in coincidence, and the one with a \qty{1}{\mm} hole (S1) may provide a veto (anti-coincidence). The trigger for the data acquisition can be either provided by a combination of scintillators or directly by one of the two DPTS.}
    \label{fig:testbeam-setup}
\end{figure}